\documentclass[a4paper,12pt, left=6cm, right=1.5cm]{thesis}

\usepackage[utf8]{inputenc}

\usepackage{blindtext}


\usepackage[onehalfspacing]{setspace}

\usepackage[ngerman]{babel}
\usepackage[T1]{fontenc}
\usepackage{amsfonts}
\usepackage{amsmath}
\usepackage{mathtools} 
\usepackage{mathabx}
\usepackage{graphicx}
\usepackage[table]{xcolor}
%\usepackage{hyperref}

\usepackage{setspace}

\usepackage{color}
\usepackage{transparent}

\usepackage{tikz}
\usetikzlibrary{positioning}
\usetikzlibrary{arrows,calc}
%\usepackage{pgfplots} % LaTeX
\usepackage{colortbl}
%\usepackage{pgfplotstable}
\usepackage{booktabs, colortbl}

\usepackage{eurosym}

\usepackage{csvsimple}

\usepackage[authoryear]{natbib}
%\bibliographystyle{apalike}
\usepackage[hidelinks]{hyperref}
\bibliographystyle{apalike}

\newcommand*{\captionsource}[2]{%
	\caption[{#1}]{%
		#1%
		\\\hspace{\linewidth}%
		\textbf{Quelle:} #2%
	}%
}

\tikzset{
%Define standard arrow tip
>=stealth',
%Define style for different line styles
help lines/.style={dashed, thick},
axis/.style={<->},
important line/.style={thick},
connection/.style={thick, dotted},
}


\begin{document}

%%% TITELSEITE %%%

\begin{center}								% Beginn einer center-Umgebung. Der Text innerhalb der center-Umgebung wird zentriert. Ansonsten wird Blocksatz verwendet
	\begin{LARGE}								% LARGE beschreibt eine Schriftgröße in Latex. Der Standard ist \normalsize. Eine übersicht der Schriftgrößen steht z.B. auf der Seite: https://www.latex-kurs.de/fragen/schriftgroesse.html 
		Faktoren des Fahrrad Verkehrs in Mannheim						% Text, der nun zentriert und in größerer Schrift geschrieben wird 
	\end{LARGE}								% Die Verwendung der Schriftgröße LARGE wird beendet. Es gilt ab jetzt wieder die normale Schriftgröße.
	
	\vspace{\fill}								% Befehl, der den vertikalen Platz (vspace) "füllt" 
	
	\begin{large}								% Der folgende Text hat nun die Schriftgröße large 
		Maximilian Samuer Weinhold\\					% Durch ein \\ wird eine neue Zeile angefangen. 
		Economics, 6. Semester\\
		505314\\
		mweinhol@uni-muenster.de\\
		
		\vspace{\fill}
		
		Hausarbeit im Rahmen des Seminars zur\\
		Analyse von Fahrrad-Verkehrsdaten\\
		Sommersemester 2021\\
		Institut für Verkehrswissenschaft\\
		Dr. Jan Wessel\\
	\end{large}
	
	\thispagestyle{empty}						% Definiert den Stil für diese Seite. empty bedeutet, dass keine Seitenzahl auf der Seite gedruckt wird. 
	
\end{center}								% Ab nun wird wieder Blocksatz verwendet

\newpage									% Seitenumbruch. Es beginnt eine neue Seite


%\chapter*{Tabellen und Abbildungen}\addchapmark{Tabellen und Abbildungen}
\onehalfspacing	
\thispagestyle{empty}	
\tableofcontents
%\begingroup
%\let\clearpage\relax
%\listoffigures
%\endgroup

%\newpage

%\listoffigures

%\newpage


%\listoffigures
%\addcontentsline{toc}{chapter}{Abbildungsverzeichnis}

\chapter{Einleitung}

\chapter{Literaturüberblick}

\chapter{Beschreibung der Daten}

\section{EcoCounter Mannheim}

\section{VRnextbike Daten}

\chapter{Vorgegebene Aufgaben}

\section{Deskriptive Analyse}

\section{Regressionsanalyse}

\chapter{Einfluss des VRnextbike Angebots}

\chapter{Fazit}


\newpage
\addcontentsline{toc}{chapter}{Literaturverzeichnis}
\bibliography{bib1}

\end{document}
